\documentclass[../../main.tex]{subfiles}

\begin{document}

\subsection{Motivation}

Each report must be created in a separate subfolder. Take the "example" folder as an example of a report (please do not change).

Here it is required to describe the problematic of the problem. Basic premises and results. It is important to indicate a list of references on this topic~\cite{AuthorYear}. All refernces put to the main.bib file.

After writing your report, include the report to main.tex similar to the sample report.

All changes are made through GitHub\footnote{\url{https://github.com/Intelligent-Systems-Phystech/GeometricDeepLearning}}. It is recommended to create a new project in overleaf by using <<Import from GitHub>> utility. And after all changes push it back to GitHub by using <<Menu -> Sync -> GitHub>>.

As an example, consider a linear regression problem.

\subsection{Problem statement}

Given a dataset of size~$m$:
\[
\label{eq:example:1}
\begin{aligned}
    \mathfrak{D} = \left\{\mathbf{x}_i, y_i\right\}_{i=1}^{m},
\end{aligned}
\]
where~$\mathbf{x}_i \in \mathbb{R}^{n}, y_i \in \mathbb{R},$ and~$n$ is a size of feature space.

\subsection{Problem solution}

The solution for the linear regression task is:
\[
\label{eq:example:2}
\begin{aligned}
   \hat{\mathbf{w}} = \arg\min_{\mathbf{w}\in \mathbb{R}^{n}} \frac{1}{m}\sum_{i=1}^{m}\left(\textbf{w}^{\mathsf{T}}\textbf{x}_i - y_i\right)^2,
\end{aligned}
\]
where~$\hat{\mathbf{w}}$ is a solution.

\subsection{Code analysis}

This section requires a description of the code. Source citation is required if this code is taken from open sources repositories.

\subsection{Experiment}

\begin{figure}[h!]
\centering
\includegraphics[width=0.6\textwidth]{figures/fig1}
\caption{Some description}
\label{fig:example:1}
\end{figure}

This section requires some experiment analysis. The chapter assumes some graphs with their descriptions~\ref{fig:example:1}.



\end{document}